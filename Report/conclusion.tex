% conclusion.tex {Conclusion}

\chapter*{Summary and Future scope}
We analysed various state of art models for Image Super Resolution task of wireless capsule endoscopy images and started working on the proposed architecture. After conduction series of experiments with different model architectures including state of art models such as SRGAN, CycleGAN, DR-Densenet and RCAN, we proposed various models including cycleCNN, cycleCNN with GRL, dense cycleCNN with GRL on different losses including MSE loss, cycle loss, adversarial loss. But this models were not able to outperform the state of art models, After gaining knowledge from all the experiments, we were able to come up with the architecture DenseNET with Channel Attention Block(DCAN) that is able to out perform all the exiting SOTA models, not just in getting best values for all the image quality assessment metrices, but also in terms of consistency. we measured model consistency through statistical parameters such as standard deviation and box plots.
The proposed model is providing better results than current state of art models DenseNet and RCAN, But it is not able to generalize over different datasets such as KID dataset and conventional endoscopy data. In future we aim to develop an deep learning based architecture that is roboust over all the datasets and we also plan to use the unsupervised approach and develop state of art models for this approach, as very limited amount of data is available for this domain.
